%source: http://tex.stackexchange.com/a/150903/23931
\documentclass{article}
\usepackage[letterpaper,margin=1in]{geometry}
\usepackage{xcolor}
\usepackage{fancyhdr}
\usepackage{tgschola} % or any other font package you like
\usepackage{verbatim}

\pagestyle{fancy}
\fancyhf{}
\fancyhead[C]{%
  \footnotesize\sffamily
  \yourname\quad
  web: \textcolor{blue}{\itshape\yourweb}\quad
  \youremail}

\newcommand{\soptitle}{Statement of Purpose}
\newcommand{\yourname}{Dante Buhl}
\newcommand{\youremail}{dbuhl8@gmail.com}
\newcommand{\yourweb}{https://dbuhl8.github.io/website/}

\newcommand{\statement}[1]{\par\medskip
  \underline{\textcolor{blue}{\textbf{#1:}}}\space
}

\usepackage[
  colorlinks,
  breaklinks,
  pdftitle={\yourname - \soptitle},
  pdfauthor={\yourname},
  unicode
]{hyperref}

\begin{document}

\begin{center}\LARGE\soptitle\\
\large of \yourname\ (Mechanical Engineering PhD applicant for Fall---2024)
\end{center}

\hrule
\vspace{1pt}
\hrule height 1pt

\bigskip

\large
%Intro
Growing up in the Greater Sacramento Area and coming from a low-income family, I had never thought I would want to become an academic or get a Ph.D. one day. And yet, here I am at 22 years of age with a drive to learn more and make new insights in fluid dynamics. I went to the University of California Santa Cruz as a first-gen student for my undergrad in Mathematics and graduated in 3 short years with a GPA of 3.74 and ``highest honors in the major''. Before my departure, I found a department doing an exciting fusion of mathematics and computer science. This was the Applied Mathematics Department (later referenced as A.M. Department), which had an amazing 4+1 master's program. This is my current academic position as I am writing this application: a graduate student in UC Santa Cruz's A.M. Department. 

%Undergrad
My undergraduate experience, although overall very strong, had a rough start. The primary reason was a result of the online medium of education in my freshman year, at the height of the pandemic. Since most of my assignments and lectures could be done at my leisure, I began working over 55 hours per week to support my family. I eventually failed a course due to this practice and had to re-evaluate my study style. Besides this, high notes of my undergraduate career were research projects I did in my last year of undergrad. The first project investigated the chaotic Lorenz ski-slope system found in Lorenz's book, \textit{The Essence of Chaos}. Our work involved reproducing Poincar\'e maps, phase portraits, and visualizations of the four-dimensional chaotic attractor. The second research project focused on numerically computing the Lyapunov dimension of three-dimensional chaotic attractors and utilized the Gram-Schmidt Ortho-normalization Process and variational methods for dynamical systems. These experiences ultimately introduced me to the complex world of numerical methods for differential equations and linear algebra, which would become the focus of my master's degree. 

I took two graduate-level fluid dynamics courses toward the end of my undergraduate career. One was an introductory course taught by Nicholas Brummell, and the other was a Geophysical fluid dynamics course taught by Chris Edwards. These courses cemented my interest in fluid dynamics and prepared me for work in that field of study. 

%Grad
Upon graduating from UC Santa Cruz in June 2023, I participated in Towson University's summer REU program in Maryland. The principal investigator for this research was Herve Nganguia, a fluid dynamicist specializing in math-bio-related problems. Our work focused on using Deep-Learning to create numerical models that complemented prior analytical work published in a paper by Nganguia concerning the propulsion efficiency of ciliated spherical ``squirmers'' in Stokes Flow. The program also focused on preparing participants in research-focused mathematical writing and communication. A notable presentation on this work is scheduled for the Joint Mathematics Meeting in San Francisco this coming January 2024. The REU ended at the beginning of August 2023, and I started a new research project in Santa Cruz. 

In my first quarter of graduate studies, I'm currently planning to complete a master's thesis on the effect of rotation on stratified turbulence in stellar fluids under the guidance of Pascale Garaud, a well-published researcher in astrophysical and geophysical fluid dynamics. This research involves Direct Numerical Simulations (DNS) and analytical work, which will append to Garaud's prior findings on stratified turbulence. Specific details include multi-scale analyses of the governing equations and an investigation of the flow regimes originating from relevant non-dimensional numbers. The DNS models are designed using MPI, a high-performance computing standard, and feature stochastic forcing methods with Gaussian Processes, Spectral Integration Methods, and triply-periodic domains. In addition to my thesis, I plan to take another course in fluid dynamics titled ``Waves, Instabilities, and Turbulence'' to further prepare for a PhD on the subject. Funding for my master's degree is secured by working as a Teaching Assistant for derivative and integral calculus courses, utilizing my background of equity-based pedagogy from prior positions with learning centers. The expected time of completion for the degree is the spring or summer of 2024. 

%Transition to future with a specific school and why I could be good there.
Moving into the future, I see myself doing well at UC San Diego's graduate program. I want to stay in California, where I was born and raised. San Diego also has a great fluids program with appealing faculty members. Sutanu Sarkar within the CFD Lab at UC San Diego is one such faculty member, specializing in computational methods for fluids problems, including turbulence, transport, and instabilities. The CFD Lab's goal of investigating natural environment flows means that my work here would readily apply to real-world problems. My numerical methods and fluid dynamics background match this program very well. Stephan Llewellyn Smith's work on convection and vortices using analytical methods is also fascinating. Since Smith works with Ocean Science, I doubt there would be any lack of GFD problems to work on with him. I've loved every encounter with GFD I've had so far, especially with waves. My background would also be a good fit for the type of work that Stephan does. Ultimately, my current advisor has recommended both professors, and I'd love to work with them as they are well-founded researchers dedicated to Fluid Dynamics. 

Another source of motivation to continue graduate studies is teaching. I've been working happily as a learning facilitator for over a year to pay for my education. So far, I've worked as Teaching Assistant for Math 19A/B and as a Peer-Group Tutor for Math 19A and Math 11A at Learning Support Services within UC Santa Cruz. Math 19A and B are paired courses on derivative and integral calculus for STEM majors, and Math11A andB are the equivalent courses for Biology majors. As part of my training at Learning Support Services, I took a course in equity-based pedagogy that focused on active learning strategies and effective peer-guided learning. The impact of such pedagogy has been shown to improve the passing rates of minority students. Because of this, I've been using the theory I learned in that course to inform my actions as a learning assistant and facilitator. Popularizing this type of pedagogy and being able to provide more accessible education to students has made this job my favorite out of all of my jobs so far. Pursuing a career that allows me to study fluid dynamics and interact with and influence the incoming generation of students is, therefore, highly appealing.

%Conclusion
Ultimately, I'm a proud and hard-working student with a passion for Fluid Dynamics. With a career in engineering and applied mathematics research in mind, a Ph.D. would further my interests and bring me into a position of being able to do advanced work in engineering fields later in life. My recent academic experiences have culminated into a well-rounded foundation for PhD research, and I look forward to navigating opportunities in the future. Thank you for considering me in your program. 

%The prospect of continuing to do advanced research in this field is highly attractive, both in the realms of numerical work and analytical work. My experience with prior research projects has helped me understand in which research situations I thrive in and those which I do not. I'm very excited to do more work in the future and find my place in a new project. Thank you for considering me in your program. 

\end{document}
