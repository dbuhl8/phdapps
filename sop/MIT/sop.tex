%source: http://tex.stackexchange.com/a/150903/23931
\documentclass{article}
\usepackage[letterpaper,margin=1in]{geometry}
\usepackage{xcolor}
\usepackage{fancyhdr}
\usepackage{tgschola} % or any other font package you like
\usepackage{verbatim}

\pagestyle{fancy}
\fancyhf{}
\fancyhead[C]{%
  \footnotesize\sffamily
  \yourname\quad
  web: \textcolor{blue}{\itshape\yourweb}\quad
  \youremail}

\newcommand{\soptitle}{Statement of Objectives}
\newcommand{\yourname}{Dante Buhl}
\newcommand{\youremail}{dbuhl8@gmail.com}
\newcommand{\yourweb}{https://dbuhl8.github.io/website/}

\newcommand{\statement}[1]{\par\medskip
  \underline{\textcolor{blue}{\textbf{#1:}}}\space
}

\usepackage[
  colorlinks,
  breaklinks,
  pdftitle={\yourname - \soptitle},
  pdfauthor={\yourname},
  unicode
]{hyperref}

\begin{document}

\begin{center}\LARGE\soptitle\\
\large of \yourname\ (Mechanical Engineering PhD applicant for Fall---2024)
\end{center}

\hrule
\vspace{1pt}
\hrule height 1pt

\bigskip

\large
%Intro
Growing up in the Greater Sacramento Area and coming from a low-income family, I had never thought I would want to become an academic or get a Ph.D. one day. And yet, here I am at 22 years of age with a drive to learn more and make new insights in fluid dynamics. I went to the University of California Santa Cruz as a first-gen student for my undergrad in Mathematics and graduated in 3 short years with a GPA of 3.74 and ``highest honors in the major''. Before my departure, I found a department doing an exciting fusion of mathematics and computer science. This was the Applied Mathematics Department (later referenced as A.M. Department), which had an amazing 4+1 master's program in Scientific Computing and Applied Mathematics. This is my current academic position as I am writing this application: a graduate student in UC Santa Cruz's A.M. Department. 

%Motivation, Undergrad, and Personal Background
My undergraduate experience, although overall very strong, had a rough start. The primary reason was a result of the online medium of education in my freshman year, at the height of the pandemic. Since most of my assignments and lectures could be done at my leisure, I began working over 55 hours weekly to support my family. I eventually failed a course due to this practice and had to re-evaluate my study style. Besides this, high notes of my undergraduate career were my fluids dynamics courses that cemented my motivation for research in fluid dynamics. The first course was an introductory course taught by Nicholas Brummell, which introduced me to the PDEs often seen in the subject. The other was a geophysical fluid dynamics course taught by Chris Edwards, which explored many geophysical phenomena such as geostrophy, waves, and Ekman dynamics. My work here ultimately inspired my choice of REU and master's research and led me to take more fluid dynamics courses in the future. 

% Expertise and reesearch
I also have experience with scientific and mathematical research. One experience was in Towson University's summer REU program in Maryland. The principal investigator for this research was Herve Nganguia, a fluid dynamicist specializing in math-bio-related problems. Our work focused on using Deep-Learning to create numerical models that complemented prior analytical work published in a paper by Nganguia concerning the propulsion efficiency of ciliated spherical ``squirmers'' in Stokes Flow. The program also focused on preparing participants in research-focused mathematical writing and communication. A notable presentation on this work is scheduled for the Joint Mathematics Meeting in San Francisco this coming January 2024. The other experience is my master's thesis. I plan to complete a master's thesis by the end of the school year on the effect of rotation on stratified turbulence in stellar fluids under the guidance of Pascale Garaud, a well-published researcher in astrophysical and geophysical fluid dynamics. This research involves Direct Numerical Simulations (DNS) and analytical work, which will append to Garaud's prior findings on stratified turbulence. Specific details include multi-scale analyses of the governing equations and an investigation of the flow regimes originating from relevant non-dimensional numbers. The DNS models are designed using MPI, a high-performance computing standard, and feature stochastic forcing methods with Gaussian Processes, Spectral Integration Methods, and triply-periodic domains. The work done on this project is highly relevant to current research methods widely used in the field

%Transition to future with a specific school and why I could be good there.
Moving into the future, I see myself fitting in well at MIT. Cambridge, Massachusetts is a fantastic place to study and gives rise to hundreds of talented academic minds every year. I've also heard fantastic things about specific fluid dynamics groups there. From my own internet searches and the recommendation of my current advisor, Tom Peacock, Anette Hosoi, and Lydia Bouroubia are faculty members that stand out above the others. These are three professors that each conduct their own fascinating research on the topic of fluid dynamics. For my Ph.D. project, I would ideally be doing a fluid dynamics project that is a good mixture of analytical and numerical work. I have a passion for both the equations seen in the subject and the methods of computation to experiment with analytical findings. Whether it is ocean and environmental dynamics, biological flows, or pathogen transport, the specializations of each faculty member are quite fascinating. I see the project I do at MIT being very applicable to real-world engineering problems and putting me on a career path in engineering and fluid dynamics research. This is exactly my desire. I believe that my previous research experience is an excellent primer for the requirements of a Ph.D. project in fluid dynamics.  

%Teaching Experience thing
Another source of motivation to continue graduate studies is teaching. I've worked happily as a learning facilitator for over a year to pay for my education. So far, I've worked as a Teaching Assistant for Math 19A/B and a Peer-Group Tutor for Math 19A and Math 11A at Learning Support Services within UC Santa Cruz. Math 19A and B are paired courses on derivative and integral calculus for STEM majors, and Math 11A and B are the equivalent courses for Biology majors. As part of my training at Learning Support Services, I took a course in equity-based pedagogy that focused on active learning strategies and effective peer-guided learning. The impact of such pedagogy has been shown to improve the passing rates of minority students. Because of this, I've been using the theory I learned in that course to inform my actions as a learning assistant and facilitator. Popularizing this type of pedagogy and being able to provide more accessible education to students has made this job my favorite out of all of my jobs so far. Pursuing a career that allows me to study fluid dynamics and interact with and influence the incoming generation of students is, therefore, highly appealing.

%Conclusion
Ultimately, I'm a proud and hard-working student with a passion for Fluid Dynamics. With a career in engineering and applied mathematics research in mind, a Ph.D. would further my interests and bring me into a position to do advanced work in engineering fields later in life. My recent academic experiences have culminated into a well-rounded foundation for PhD research, and I look forward to navigating opportunities in the future. Thank you for considering me in your program. 

\end{document}
