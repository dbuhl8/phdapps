%source: http://tex.stackexchange.com/a/150903/23931
\documentclass{article}
\usepackage[letterpaper,margin=1in]{geometry}
\usepackage{xcolor}
\usepackage{fancyhdr}
\usepackage{tgschola} % or any other font package you like
\usepackage{verbatim}

\pagestyle{fancy}
\fancyhf{}
\fancyhead[C]{%
  \footnotesize\sffamily
  \yourname\quad
  web: \textcolor{blue}{\itshape\yourweb}\quad
  \youremail}

\newcommand{\soptitle}{Diversity Statement}
\newcommand{\yourname}{Dante Buhl}
\newcommand{\youremail}{dbuhl8@gmail.com}
\newcommand{\yourweb}{https://dbuhl8.github.io/website/}

\newcommand{\statement}[1]{\par\medskip
  \underline{\textcolor{blue}{\textbf{#1:}}}\space
}

\usepackage[
  colorlinks,
  breaklinks,
  pdftitle={\yourname - \soptitle},
  pdfauthor={\yourname},
  unicode
]{hyperref}

\begin{document}

\begin{center}\LARGE\soptitle\\
\large of \yourname\ (Applied Math PhD applicant for Fall---2024)
\end{center}

\hrule
\vspace{1pt}
\hrule height 1pt

\bigskip

\large

Diversity is an important part of academic communities in the modern age. As part of my life, I've always been cognizant of it. I grew up in a very white part of California. So white, in fact, that only 1\% of students at my school were black. I'm half-Peruvian and half-white first generation student myself, and my family was considered a low-income family. While there are a lot of details which describe my own experiences as a minority, I prefer to think about how I use my diversity to influence other people's experiences. 

One of my favorite ways of giving back as a diverse student was by working as a tutor and now as a teaching assistant. Growing up with busy parents who never finished college, there was only so much help I could get from them, and generally I didn't have the resources to get other help if I needed it. Later in life, I realized that I was very lucky to have learned math and science as well as I did. Pedagogy studies have shown that underrepresented minority students have a much higher chance of failing STEM courses when they don't have access to resources that allow them to engage in active learing environments. As a learning facilitator, I practice equity-minded pedagogy which is focused on providing equally effective active learning experiences to a large group of students. This is a resource that I can now provide to students who may not have had equal experiences in their previous education. I find it very fitting, mostly because I do love helping students understand Calculus. Calculus is easily my favorite subject, but also a subject that many people turn away from after their first interaction with it. Ideally, by being mindful of the way I help students engage with mathematics, I will make the field of mathematics more welcoming for diverse students. 

This is my goal as an educator, and perhaps the most potent lens which I engage with diversity. Not simply by being a minority or a first generation student myself, but by helping mathematics become more diverse in the minds it can reach. Too many students have a negative mental image of math in their heads. I think the best way to change the general lack of diverity in the field of mathematics is to diversify the mental images in peoples' minds. 

\end{document}
