%source: http://tex.stackexchange.com/a/150903/23931
\documentclass{article}
\usepackage[letterpaper,margin=1in]{geometry}
\usepackage{xcolor}
\usepackage{fancyhdr}
\usepackage{tgschola} % or any other font package you like
\usepackage{verbatim}

\pagestyle{fancy}
\fancyhf{}
\fancyhead[C]{%
  \footnotesize\sffamily
  \yourname\quad
  web: \textcolor{blue}{\itshape\yourweb}\quad
  \youremail}

\newcommand{\soptitle}{Personal Statement}
\newcommand{\yourname}{Dante Buhl}
\newcommand{\youremail}{dbuhl8@gmail.com}
\newcommand{\yourweb}{https://dbuhl8.github.io/website/}

\newcommand{\statement}[1]{\par\medskip
  \underline{\textcolor{blue}{\textbf{#1:}}}\space
}

\usepackage[
  colorlinks,
  breaklinks,
  pdftitle={\yourname - \soptitle},
  pdfauthor={\yourname},
  unicode
]{hyperref}

\begin{document}

\begin{center}\LARGE\soptitle\\
\large of \yourname\ (Mechanical Engineering PhD applicant for Fall---2024)
\end{center}

\hrule
\vspace{1pt}
\hrule height 1pt

\bigskip

\large
%Intro - 
Growing up in the Greater Sacramento Area and coming from a low-income family, I had never thought I would want to become an academic or get a PhD one day. And yet, here I am at 22 years of age with a drive to learn more and make new insights in fluid dynamics. I went to the University of California Santa Cruz as a first-gen student for my undergrad in Mathematics,  and graduated in 3 short years with a GPA of 3.74 and highest honors in the major. Before my departure, I found a department that was doing a really interesting fusion of mathematics and computer science. This was, of course, the Applied Mathematics Department (later referenced as A.M. Department), which happened to have an fantastic 4+1 master's program. This is my current academic position as I am writing this: a graduate student in UC Santa Cruz's A.M. Department. 

%Undergrad (eliminate some of this)
My undergraduate experience, although overall very strong, had a rough start. The primary reason was a result of the online medium of education in my freshman year, at the height of the pandemic. Since most of my assignments and lectures could be done at my leisure, I began working over 55 hours per week to support my family. I eventually failed a course due to this practice and had to re-evaluate my study style. Despite this, I still graduated early and managed to get into the Applied Mathematics Masters Program. Toward the end of my undergraduate career, I found the fluid dynamics group at UCSC and did some research projects on Dynamical Systems. From the fluid dynamics courses, I learned that I really enjoyed studying PDE's and fluids problems. They were stimulating and pushed me to learn, unlike many classes before it. Then, from my research projects on dynamical systems and chaos, I found that the open-ended nature of the projects were highly appealing, and I found myself more invested in that type of work than course based studies. These experiences ultimately inspired me to continue with more research projects and begin studying the complex world of numerical methods for differential equations and linear algebra, which would become the focus of my master's degree. Now, I want to continue this trend with a PhD project focused on Fluid Dynamics.

% Add a section about TAing and teaching experience (desire to keep teaching and learning)
Another source of motivation to continue graduate studies is teaching. I've been working happily as a learning facilitator for over a year now to pay for my education. So far, I've worked as a Peer-Group Tutor for Math 19A and Math 11A at Learning Support Services within UC Santa Cruz and as a Teaching Assistant for Math 19A and Math 19B. Math 19A/B are paired courses on derivative and integral calculus courses for STEM majors, and Math11A/B are the equivalent courses for Biology majors. As part of my training at Learning Support Services, I took a course in equity-based pedagogy, which focused on active learning strategies and effective peer-guided learning. The impact of such pedagogy has been shown to improve the passing rates of minority students. Because of this, I've been using the theory I learned in that course to inform my actions as a learning assistant and facilitator. Popularizing this type of pedagogy and being able to provide more accessible education to students has made this job my favorite out of all of my jobs so far. Pursuing a career that allows me to study and pursue fluid dynamics while also allowing me to interact with and influence the incoming generation of students is, therefore, highly appealing to me. 

%Conclusion (change this to talk more aobut doing graduate work)
Ultimately, I'm a proud and hard-working student who has, within the last year, developed a passion for working with Navier-Stokes and Fluid Dynamics. My recent academic experiences have culminated into a well-rounded foundation for PhD research and developed a thirst to try to answer open-ended questions. For these reasons, I'm applying to graduate study at UC Berkeley. 

%The prospect of continuing to do advanced research in this field is highly attractive, both in the realms of numerical work and analytical work. My experience with prior research projects has helped me understand in which research situations I thrive in and those which I do not. I'm very excited to do more work in the future and find my place in a new project. Thank you for considering me in your program. 

\end{document}
