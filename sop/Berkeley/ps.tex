%source: http://tex.stackexchange.com/a/150903/23931
\documentclass{article}
\usepackage[letterpaper,margin=1in]{geometry}
\usepackage{xcolor}
\usepackage{fancyhdr}
\usepackage{tgschola} % or any other font package you like
\usepackage{verbatim}

\pagestyle{fancy}
\fancyhf{}
\fancyhead[C]{%
  \footnotesize\sffamily
  \yourname\quad
  web: \textcolor{blue}{\itshape\yourweb}\quad
  \youremail}

\newcommand{\soptitle}{Personal Statement}
\newcommand{\yourname}{Dante Buhl}
\newcommand{\youremail}{dbuhl8@gmail.com}
\newcommand{\yourweb}{https://dbuhl8.github.io/website/}

\newcommand{\statement}[1]{\par\medskip
  \underline{\textcolor{blue}{\textbf{#1:}}}\space
}

\usepackage[
  colorlinks,
  breaklinks,
  pdftitle={\yourname - \soptitle},
  pdfauthor={\yourname},
  unicode
]{hyperref}

\begin{document}

\begin{center}\LARGE\soptitle\\
\large of \yourname\ (Mechanical Engineering PhD applicant for Fall---2024)
\end{center}

\hrule
\vspace{1pt}
\hrule height 1pt

\bigskip

\large
%Intro - 
Growing up in the Greater Sacramento Area and coming from a low-income family, I had never thought that I would want to become an academic, or at least get a PhD one day. And yet, here I am at 22 years of age with a drive to learn more and make new insights in the field of fluid dynamics. I went to the University of California Santa Cruz as a first-gen student for my undergrad in Mathematics and graduated in 3 short years with a GPA of 3.74 and highest honors in the major. Before my departure, I found a department that was doing a really interesting fusion of mathematics and computer science. This was, of course, the Applied Mathematics Department (later referenced as A.M. Department), which happened to have an amazing 4+1 master's program. This is my current academic position as I am writing this: a graduate student in UC Santa Cruz's A.M. Department. 

%Undergrad (eliminate some of this)
My undergraduate experience, although overall very strong, had a rough start. This was primarily a result of the online medium of education in my freshman year, at the height of the pandemic. Since most of my assignments and lectures could be done at my leisure, I began working upwards of 55 hours per week in order to support my family. I eventually failed a course due to this practice, and had to re-evaluate my study style. Despite this, I still graduated in 3 years and managed to get into the Applied Mathematics masters program. Towards the end of my undergraduate career, I did some research projects in the field of Applied Math. One project was an investigation of the Lorenz Ski-Slope system published in Lorenz's book, \textit{The Essence of Chaos}. The work involved reproducing poincare maps, phase portraits, and even visualizations of the four-dimensional chaotic attractor found in the book. The second was for the Mathematics Department's Directed Reading Program in the Winter Quarter of 2023. This research was focused on numerical methods for computing the lyapunov dimension of three dimensional chaotic attractors, and utilized the Gram-Schmidt Ortho-normalization Process and variational methods for dynamical systems. These experiences ultimately inspired me to begin studying the complex world of numerical methods for differential equations and linear algebra which would become the focus of my master's degree. 

% Add a section about TAing and teaching experience (desire to keep teaching and learning)
Some of my motivation for graduate studies is also teaching. I've been working as a learning facilitator for over a year now in order to pay for my educaiton, and it is often a very enriching experience. So far, I've worked as a Peer-Group Tutor for Math 19A and Math 11A at Learning Support Services within UC Santa Cruz and as a Teaching Assistant for Math 19A and Math 19B. Math 19A/B are paired derivative and integral calculus courses for STEM majors, and Math11A/B are the equivalent for Biology majors. As part of my training at Learning Support services, I took a course in equity-based pedagogy which focused on active learning strategies and effective peer-guided learning. The impact of such pedagogy has shown to improve the passing rates of minority students. I've been using the theory I learned in that course to inform my actions as a learning assistant and facilitator. So far my position as a learning facilitator both as a tutor and TA have been my favorite jobs so far. Pursueing a career which allows me to study and pursue fluid dynamics while also allowing me to interact with and influence the incoming generation of students is highly appealing to me. 

%Conclusion (change this to talk more aobut doing graduate work)
Ultimately, I'm a proud and hard-working student who has, within the last year, developed a passion for working with Navier-Stokes and Fluid Dynamics. My recent academic experiences have culminated into a well rounded foundation for PhD research and I look forward to navigating opportunities in the future. 

%The prospect of continuing to do advanced research in this field is highly attractive, both in the realms of numerical work and analytical work. My experience with prior research projects has helped me understand in which research situations I thrive in and those which I do not. I'm very excited to do more work in the future and find my place in a new project. Thank you for considering me in your program. 

\end{document}
