%source: http://tex.stackexchange.com/a/150903/23931
\documentclass{article}
\usepackage[letterpaper,margin=1in]{geometry}
\usepackage{xcolor}
\usepackage{fancyhdr}
\usepackage{tgschola} % or any other font package you like
\usepackage{verbatim}

\pagestyle{fancy}
\fancyhf{}
\fancyhead[C]{%
  \footnotesize\sffamily
  \yourname\quad
  web: \textcolor{blue}{\itshape\yourweb}\quad
  \youremail}

\newcommand{\soptitle}{Personal Statement}
\newcommand{\yourname}{Dante Buhl}
\newcommand{\youremail}{dbuhl8@gmail.com}
\newcommand{\yourweb}{https://dbuhl8.github.io/website/}

\newcommand{\statement}[1]{\par\medskip
  \underline{\textcolor{blue}{\textbf{#1:}}}\space
}

\usepackage[
  colorlinks,
  breaklinks,
  pdftitle={\yourname - \soptitle},
  pdfauthor={\yourname},
  unicode
]{hyperref}

\begin{document}

\begin{center}\LARGE\soptitle\\
\large of \yourname\ (Applied Math PhD applicant for Fall---2024)
\end{center}

\hrule
\vspace{1pt}
\hrule height 1pt

\bigskip

\large
%Intro
Growing up in the Greater Sacramento Area and coming from a low-income family, I had never thought that I would want to become an academic, or at least get a PhD one day. And yet, here I am at 22 years of age with a drive to learn more and make new insights in the field of fluid dynamics. I went to the University of California Santa Cruz as a first-gen student for my undergrad in Mathematics and graduated in 3 short years with a GPA of 3.74 and highest honors in the major. Before my departure, I found a department that was doing a really interesting fusion of mathematics and computer science. This was, of course, the Applied Mathematics Department (later referenced as A.M. Department), which happened to have an amazing 4+1 master's program. This is my current academic position as I am writing this: a graduate student in UC Santa Cruz's A.M. Department. 

%Undergrad (eliminate some of this)
My undergraduate experience, although overall very strong, had a rough start. This was primarily a result of the online medium of education in my freshman year, at the height of the pandemic. Since most of my assignments and lectures could be done at my leisure, I began working upwards of 55 hours per week in order to support my family. I eventually failed a course due to this practice, and had to re-evaluate my study style. Besides this, some high notes of my undergraduate career were research projects I did in my last year of undergrad. The first project was an investigation of the Lorenz Ski-Slope system published in Lorenz's book, \textit{The Essence of Chaos}. Our work involved reproducing poincare maps, phase portraits, and even visualizations of the four-dimensional chaotic attractor found in the book. The second was for the Mathematics Department's Directed Reading Program in the Winter Quarter of 2023. This research was focused on numerical methods for computing the lyapunov dimension of three dimensional chaotic attractors, and utilized the Gram-Schmidt Ortho-normalization Process and variational methods for dynamical systems. These experiences ultimately introduced me to the complex world of numerical methods for differential equations and linear algebra which would become the focus of my master's degree. 



% Add a section about TAing and teaching experience (desire to keep teaching and learning)
Currently, I am in my first quarter of graduate studies and plan to complete a master's thesis on the effect of rotation on stratified turbuluence in stellar fluids under the guidance of Pascale Garaud, a well-published researcher in astrophysical and geophysical fluid dynamics. This research involves both Direct Numerical Simulations (abbreviated DNS) and analytical work that will apppend to Garaud's prior findings on stratified turbulence. Some details of this work includes multi-scale analysis of the governing equations and investigation of the regimes that originate from relevant non-dimensional numbers. Features of the DNS design, in this project include stochastic forcing methods with Gaussian Processes, Spectral Integration Methods, and tripply-periodic domains. In addition to my thesis, I plan to take another course in fluid dynamics titled, ``Waves, Instabilities, and Turbuluence'', in order to prepare for a PhD on the subject. I will also be working as a Teaching Assistant in order to fund my degree, gain teaching experience, and support lower division mathematics courses within the deparment. The expected time of completion for the degree is the spring or summer of 2024. 


%Conclusion (change this to talk more aobut doing graduate work)
Ultimately, I'm a proud and hard-working student who has, within the last year, developed a passion for working with Navier-Stokes and Fluid Dynamics. My recent academic experiences have culminated into a well rounded foundation for PhD research and I look forward to navigating opportunities in the future. 

%The prospect of continuing to do advanced research in this field is highly attractive, both in the realms of numerical work and analytical work. My experience with prior research projects has helped me understand in which research situations I thrive in and those which I do not. I'm very excited to do more work in the future and find my place in a new project. Thank you for considering me in your program. 

\end{document}
