%source: http://tex.stackexchange.com/a/150903/23931
\documentclass{article}
\usepackage[letterpaper,margin=1in]{geometry}
\usepackage{xcolor}
\usepackage{fancyhdr}
\usepackage{tgschola} % or any other font package you like
\usepackage{verbatim}

\pagestyle{fancy}
\fancyhf{}
\fancyhead[C]{%
  \footnotesize\sffamily
  \yourname\quad
  web: \textcolor{blue}{\itshape\yourweb}\quad
  \youremail}

\newcommand{\soptitle}{Statement of Purpose}
\newcommand{\yourname}{Dante Buhl}
\newcommand{\youremail}{dbuhl8@gmail.com}
\newcommand{\yourweb}{https://dbuhl8.github.io/website/}

\newcommand{\statement}[1]{\par\medskip
  \underline{\textcolor{blue}{\textbf{#1:}}}\space
}

\usepackage[
  colorlinks,
  breaklinks,
  pdftitle={\yourname - \soptitle},
  pdfauthor={\yourname},
  unicode
]{hyperref}

\begin{document}

\begin{center}\LARGE\soptitle\\
\large of \yourname\ (Mechanical Engineering PhD applicant for Fall---2024)
\end{center}

\hrule
\vspace{1pt}
\hrule height 1pt

\bigskip

\large
%Intro - Why I want to go graduate study on fluid dynamics
My goals for my graduate studies are ultimately pursue a doctoral degree in the field of mechanical engineering with a specific focus on the field of fluid dynamics. I've had a keen interest in the topic for several years now, but my endeavors into the subject have only began in the last two years with fluid dynamics courses, and two research experiences. 

%Experience with and Aptitude for Fluid Dynamics (try to condense this)
I did my undergrad in Pure Mathematics at UC Santa Cruz starting in 2020. In my 3rd and last year of undergrad, I met some of the Fluid Dynamics faculty at UC Santa Cruz and soon became very invested in the program that they had. Within the next two quarters of study, I took two graduate level fluid dynamics courses. One was an introductory course taught by Nicholas Brummell from the A.M. Department, and the other by Chris Edwards from the Ocean Sciences department and focused on Geophysical Fluid Dynamics. In the near future, I also plan to take another course, AM 227: "Waves, Instabilities, and Turbulence" in the Winter of 2024. These courses cemented my interest in the general subject of fluid dynamics while also preparing me for work in the field of study. 

%Turn these two paragraphs into a paragraph about research experience in Fluids
Besides coursework, I also have some research experience with fluid dynamics. Upon graduating with my BA in mathematics in June 2023, I participated in a summer REU program at Towson University. The principal investigator for this research was Herve Nganguia, a fluid dynamicist who specializes in math-bio related problems. Our work was focused on using Deep-Learning to create numerical models which complemented prior analytical work published in a paper by Nganguia concerning the propulsion efficiency of ciliated spherical ``squirmers'' in Stokes Flow. As a result, our neural networks were trained on low Reynolds number Navier-Stokes in various coordinate systems, including spherical and spheroidal coordinates. The other major research experience is in conjunction with my master's degree at UC Santa Cruz. For my thesis, I'm currently working on a paper about the effect of rotation on stratified turbuluence in stellar fluids under the guidance of Pascale Garaud, a well-published researcher in astrophysical and geophysical fluid dynamics. This research involves Direct Numerical Simulations (abbreviated DNS) and analytical work that will apppend to Garaud's prior findings on stratified turbulence. Some details of this work includes multi-scale analysis of the governing equations and investigation of the regimes that originate from relevant non-dimensional numbers. The DNS models are designed using MPI, a high-performance computing standard, and feature stochastic forcing methods with Gaussian Processes, Spectral Integration Methods, and tripply-periodic domains. While, I've done other research projects on the topic of dynamical systems (more information on which can be found on my CV and website), these two projects detail most of my work in the field of fluid dynamics so far. 

%This program was also focused on preparing participants in research-focused mathematical work and featured several presentation opportunities. The most notable presentation on this work is scheduled to be at the Joint Mathematics Meeting in San Francisco this coming January 2024. This program gave me some hands on experience with budding problems in the community as well as familiarized myself with the focus needed for research. 

%Why I want to do it at berkeley (try to expand this)
Moving into the future, I could see myself fitting in very well at UC Berkeley. I very much want to remain in California, the state I grew up in and Berkeley is not very far from where I live now in Santa Cruz. Berkeley is also known to be a great institution for Fluid Dynamics with a large list of faculty to support the field of study. Of the many names at Berkeley, some of those that stand out are Alexis Kaminski, Simo M$\ddot{a}$kiharju, and Jon Wilkening. I had heard of these professors from my current advisor, and looking at their research and publications I think they might be great fits. Kaminski was part of a really interesting paper submitted last year on the exhibition of sensitive dependence on initial conditions in several stratified turbulence models. With my experience with stratified turbulence and GFD, I think I would do well in Kaminski's group. As for Simo M$\ddot{a}$kiharju, his involvement with the FLOW Lab at UC Berkeley is a strong interest of mine.  

%MAYBE CHANGE THIS
Ultimately, I'm a proud and hard-working student who has, within the last year, developed a passion for working with Navier-Stokes and Fluid Dynamics. My recent academic experiences have culminated into a well rounded foundation for PhD research and I look forward to navigating opportunities in the future. 


\end{document}
