%source: http://tex.stackexchange.com/a/150903/23931
\documentclass{article}
\usepackage[letterpaper,margin=1in]{geometry}
\usepackage{xcolor}
\usepackage{fancyhdr}
\usepackage{tgschola} % or any other font package you like
\usepackage{verbatim}

\pagestyle{fancy}
\fancyhf{}
\fancyhead[C]{%
  \footnotesize\sffamily
  \yourname\quad
  web: \textcolor{blue}{\itshape\yourweb}\quad
  \youremail}

\newcommand{\soptitle}{Personal Statement}
\newcommand{\yourname}{Dante Buhl}
\newcommand{\youremail}{dbuhl8@gmail.com}
\newcommand{\yourweb}{https://dbuhl8.github.io/website/}

\newcommand{\statement}[1]{\par\medskip
  \underline{\textcolor{blue}{\textbf{#1:}}}\space
}

\usepackage[
  colorlinks,
  breaklinks,
  pdftitle={\yourname - \soptitle},
  pdfauthor={\yourname},
  unicode
]{hyperref}

\begin{document}

\begin{center}\LARGE\soptitle\\
\large of \yourname\ (Applied Math PhD applicant for Fall---2024)
\end{center}

\hrule
\vspace{1pt}
\hrule height 1pt

\bigskip

\large
%Intro
Growing up in the Greater Sacramento Area and coming from a low-income family, I had never thought that I would want to become an academic, or at least get a PhD one day. And yet, here I am at 22 years of age with a drive to learn more and make new insights in the field of fluid dynamics. I went to the University of California Santa Cruz as a first-gen student for my undergrad in Mathematics and graduated in 3 short years with a GPA of 3.74 and highest honors in the major. Before my departure, I found a department that was doing a really interesting fusion of mathematics and computer science. This was, of course, the Applied Mathematics Department (later referenced as A.M. Department), which happened to have an amazing 4+1 master's program. This is my current academic position as I am writing this: a graduate student in UC Santa Cruz's A.M. Department. 

%Undergrad
My undergraduate experience, although overall very strong, had a rough start. This was primarily a result of the online medium of education in my freshman year, at the height of the pandemic. Since most of my assignments and lectures could be done at my leisure, I began working upwards of 55 hours per week in order to support my family. I eventually failed a course due to this practice, and had to re-evaluate my study style. Besides this, some high notes of my undergraduate career were my Directed Reading Program Presentation in the Mathematics Department and a small research project for an ``Applied Dynamical Systems'' course I took in Fall 2022. The former was a presentation based on a dynamical systems research project I completed in the Winter Quarter of 2023 under the mentorship of John Pelias, a Mathematics PhD student at UC Santa Cruz. This research was focused on numerical methods for computing the fractal dimension of three dimensional chaotic attractors according to the various definitions of fractal dimension. The latter project was an investigation of the Lorenz Ski-Slope system published in Lorenz's book, \textit{The Essence of Chaos}. Our work involved reproducing poincare maps, phase portraits, and even visualizations of the four-dimensional chaotic attractor found in the book. Both research projects relied on Matlab programming and deepened my knowledge of dynamical systems and numerical methods (more information about these projects can be found on my website linked above). 

Towards the end of my undergraduate career, I took two graduate level fluid dynamics courses. One was an introductory course taught by Nicholas Brummell from the A.M. Department, and the other by Chris Edwards from the Ocean Sciences department and which focused on Geophysical Fluid Dynamics. These courses cemented my interest in the general subject of fluid dynamics while also preparing me for work in the field. 

%Grad
Upon graduating from UC Santa Cruz in June 2023, I participated in a summer REU program at Towson University in Maryland. The principal investigator for this research was Herve Nganguia, a fluid dynamicist who specializes in math-bio related problems. Our work was focused on using Deep-Learning to find a numerical model which complemented prior analytical work published in a paper by Nganguia concerning the propulsion efficiency of ciliated spherical ``squirmers'' in Stokes Flow. This program was also focused on preparing participants in research-focused mathematical work and featured several presentation opportunities. An example being a scheduled presentation at the Joint Mathematics Meeting in January 2024 based on our research. The REU ended at the beginning of August, and I soon after started a new research project in Santa Cruz. 

Currently, I am in my first quarter of graduate studies and plan to complete a master's thesis on the effect of rotation on stratified turbuluence in stellar fluids under the guidance of Pascale Garaud, a well-published researcher in astrophysical and geophysical fluid dynamics. This research is a mix of Direct Numerical Simulations and analytical work which utilizes High-Performance Computing resources such as the San Diego Supercomputer Center and will use Garaud's prior findings. In addition to my thesis, I plan to take another course in fluid dynamics titled, ``Waves, Instabilities, and Turbuluence'', in order to prepare for a PhD on the subject. I will also be working as a Teaching Assistant in order to fund my degree, gain teaching experience, and support lower division mathematics courses within the deparment. The expected time of completion for the degree is the spring or summer of 2024. 

%Conclusion
Ultimately, I'm a proud and hard-working student who has, within the last year, developed a passion for working with Navier-Stokes and Fluid Dynamics. The prospect of continuing to do advanced research in this field is highly attractive, both in the realms of numerical work and analytical work. My experience with prior research projects has helped me understand in which research situations I thrive in and those which I do not. I'm very excited to do more work in the future and find my place in a new project. Thank you for considering me in your program. 


%Now, a year later, I'm in the first year of my accelerated masters program working on a masters thesis on Stratified Turbulence in Stellar Flows with Pascale Garaud. I have already taken 2 graduate level Fluid Dynamics courses at UCSC, and spent a summer in Towson's REU program with Herve Nganguia studying numerical models of propulsion in fluids at micro scales using machine learning. It is sort of amazing how fast your life can change when you find something that really sparks your interest like that. And yet, I feel I'm still missing something. This year will end before I know it and I've only just started my exploration into Fluid Dynamics and Applied Math. There are simply more classes I want to take, more professors that I want to meet, and potentially more subjects to find and become enamored with.
 
%Personal experience with these subjects have been developed in my work with Pascale Garaud in our exploration of Stably Stratified Turbulence generated with rotation and stochastic forcing in Stellar Fluids. This work is focused on using a HPC fortran script, called PADDI, which has been utilized on several supercomputers, and by Pascale and myself, most recently on San Diego's Supercomputer Cluster named Expanse. PADDI is a code which relies on a spectral decomposition of Navier-Stokes terms and proceeds to perform discretized integration using several popular numerical methods techniques. My work has been focused on developing a stochastic forcing process compatible with the PADDI's discretization structure, and which relies on Gaussian Random Processes. 

%My other research experience over the summer of 2023 with Herve Nganguia and the rest of our research group (consisting of Garrett Hauser, Kristin Lloyd, Jazmin Sharp, and Samuel Armstrong), was focused on using a deep-learning python toolbox, DeepXDE, in order to attempt to reproduce an analytical result found in one of Nganguia's papers. This project was specifically dependent on GPU accelerated deep-learning using a Physics Informed Neural Network, PINN for short. Our experimentation relied on the attunement and experimentation with the ``hyperparameters'' of the neural network, and methods of boundary condition defintion in order to optimize the performance of the network. The specific PDE we used was the Navier-Stokes equations in spherical and spheroidal coordinate systems with the inclusion of the Boussinesque Approximation in a low Reynolds Number environment. Due to the scaling arguments of the problem, we were able to use the continuity equation to reduce Navier-Stokes equations even further by simplifying the diffusivity term expansion.

%Besides my direct research experience in the field of fluid dynamics, I've completed several courses which would supplement my investigation into the field greatly. Those being ``Intro Fluid Dynamics'' and ``Geophysical Fluid Dynamics'', and I plan to take ``Waves, Instabilities, and Turbuluence in Fluids'' this Winter. My masters program at UC Santa Cruz in Scientific Computing and Applied Mathematics, SciCAM for short, also has a curriculum which is focused on numerical methods for linear algebra, differential equations, and high performance computing. All of which are very focused in further developing my ability to become a scientific researcher in the world of Direct Numerical Simulations and Fluid Dynamics. 
%Conc
%Going into the future, I want to continue this line of work in the realm of DNS in order to study fluid dynamics as it is the perfect fusion of mathematics and computing in an interesting and seemingly endless field. The prospect of a physical lab to conduct fluids experiments is also highly alluring as I've seen smaller practical demonstrations which were highly motivating. Ultimately, the pursuit of knowledge in this field is all I desire and am prepared to dedicate several years of my life to. With my current and planned future experience, I will be a strong PhD applicant with a good foundation of previous research and expertise to start a long-term project. 

%In the last two years, I've found a community of scholars that I really connect with and inspire me to do work that I had never considered before. I was raised to be an aspiring mathematician but I always had a knack for computers and loved the concept of making your own code. When I came to the Univeristy of California Santa Cruz and realized that most of the courses offered by the mathematics department weren't as glamourous as I thought they would be. It wasn't that the concept of Real Analysis, for example, was uninteresting to me. It was that I felt that I could not have done something with Real Analysis that somebody had not already done. The Art of Mathematics is thousands of years old, and that is part of what makes it so beautiful. It is in my favorite sense, the language of change, the very fabric of what our universe is made of. The community that I found, was a community of mathematicians who attemped to push the boundaries of known mathematics with the advent of new technologies. These mathematicians work in many academic departments of many different names, but I like to call them Applied Mathematicians. Applied Mathematics is a broad and ever-widening field, but most importantly it attempts to leave no questions unanswered.

%I grew up on the edge of the Greater Sacramento Area in a town called El Dorado Hills, where its always a little hotter and drier than you would like, and there isn't all that much to do. As a result, I was always sort of bored; school never quite needed my full attention to pass, and I was without many hobbies. My parents weren't very wealthy either, and divorced soon after we lost our house in the Recession. Since then my mother has had to work 2 full-time jobs to provide for the family, and I've always followed that model been a very driven and busy person. Working throughly for hours on a specific thing is something I've never had a problem with and lead me to love math in my own way. Given a fixed period of time, I could obsess over a new set of mathematical ideas. Every time a new subject was taught, if it wasn't terribly proof-heavy, I would consume it rather rapidly. Moreover, I never found it sufficient to just learn a formula or concept, I had to also know why it worked. When I learned about the Quadratic Formula, I had to try proving it myself, my teacher guided me through the proof after class per my inquiry. And once I found calculus, it seemed a limitless world had been opened to me and I decided I wanted to study Mathematics in college at UC Santa Cruz.

%By the middle of my undergrad in Mathematics at UCSC, I found myself somewhat without purpose. I had been studying math so long, just because I found it pleasant to study math, that I sort of forgot what I wanted to do with my life. I also realized that knowing how to compute some integrals wasn't what made you the coolest mathematician ever. I wanted something to specialize in, and something to be the best at. I remember looking at courses offered that would satisfy my major requirements and feeling so bored; why was this upper-division course just called ``Algebra''? It was then I realized that Mathematics was a subject so tedious and convuluted that it had lost touch with the real world. 

%That same day, I saw the course offering ``Introduction to Fluid Dynamics'' from the Applied Math department, and I became wonderfully curious.  Curious enough such that I realized I wanted to make a change. I took a Dynamical Systems class in Fall 2022 from the Applied Math department as it satisfied a requirement in my current major, and I fell in love with the concept. Proofs and specificity were still part of the subject, but the air of the subject was different. There was a freedom in that course which was never really present in any of the math classes I took. Moreover, the professor for that very course happened to be a part of the Fluids group. Very suddenly, my prior wonder become something tangible, it became conversations. Before that quarter ended, those conversations turned into an application, and I applied to the department's masters program and enrolled in the graduate level dynamical systems course. 



\end{document}
